\section[PKI and blockchains]{Background on public key infrastructure (PKI) and blockchains}


\begin{frame}{Public-key encryption}
\begin{center}
\begin{tikzpicture}
[node distance=2cm and 0.5cm]
% 'd' for dummies
% upper level: r  sk  d pk d5
% lower level: d2 d3 ct d4 pt s

% lower
\node[label=above:{Recipient}](r) {{\LARGE \faUser}};

\node[right = of r.east] (d3) {decrypt};

\node[right = of d3.east, label=below:{ciphertext}] (ct) {{\Huge \faFileText}};

\node[right = of ct.east] (d4) {encrypt};

\node[right = of d4.east, label=below:{plaintext}] (pt) {{\Huge \faFileTextO}};

\node[right = of pt.east, label=above:{Sender}](s) {{\LARGE \faUser}};

% upper

\node[above = of d3.north, label=above: {secret key}] (sk) {{\Huge \red{\faKey}}};

\node[above = of d4.north, label=above: {public key}] (pk) {{\Huge \olive{\faKey}}};


%\node[](r) {Recipient:};
%
%\node[right = of r.east, label=above: {secret key}] (sk) {{\Huge \red{\faKey}}};
%
%\node[right = of sk.east] (d) {};
%
%\node[right = of d.east, label=above: {public key}] (pk) {{\Huge \olive{\faKey}}};
%
%\node[right = of pk.east] (d5) {};

%horizontal upper
%\draw[->] (r) to (sk);

%vertical
\draw[->] (pk) to (d4);
\draw[->] (sk) to (d3);
%\draw[->] (d2) to (r);

%horizontal lower
\draw[-] (s) to (pt);
\draw[-] (pt) to (d4);
\draw[->] (d4) to (ct);
\draw[-] (ct) to (d3);
\draw[->] (d3) to (r);

\end{tikzpicture}
\end{center}

%{\large test test}
%
%{\LARGE test test}
%
%{\LARGE \faKey}
\end{frame}

\begin{frame}{Public key infrastructure}

\begin{exampleblock}{Public key infrastructure (PKI)}
\begin{itemize}
\item 
%\textbf{Public key infrastructure (PKI)}:
A set of roles
%, policies,
and procedures
%needed to
%create, manage, distribute, use, store, and revoke
%digital certificates and manage public-key encryption.
ensuring
secure distribution of public keys.
\item Based on \textbf{digital certificates}.
\end{itemize}
\end{exampleblock}

%'digital certificate = signature
%binding an entity to some public key'

%\vspace{2mm}
%
%PKI involves
%\begin{itemize}
%\item public key cryptography,
%\item digital certificates,
%\item digital identity management.
%\end{itemize}

\end{frame}

\begin{frame}{Digital certificate}
%based on https://tex.stackexchange.com/a/8935
\begin{center}
\begin{tikzpicture}[mymatrix/.style={matrix of nodes, nodes=typetag, row sep=1em},
  mycontainer/.style={draw=gray, inner sep=1ex},
  typetag/.style={draw=gray, inner sep=1ex, anchor=west},
  title/.style={draw=none, color=gray, inner sep=0pt}
  ]
  \matrix[mymatrix] (user) {
    |[title]|Entity's data (ED)\\
    unique identifier\\
    public key $\mathtt{pk}_{\mathtt{E}}$ \\
    ... \\
  };
  
  \node[right=of user.east] (fake) {};
  
  \node[above = of fake] (fakeN) {$\mathtt{sk}_{\mathtt{CA}}$};
  
  \node[draw,rectangle,
fill=white, above = of fakeN](ca) {Certificate authority (CA)};
  
    \matrix[mymatrix, right=of fake.east, matrix anchor=west] (cert) {
    |[title]|Digital certificate\\
    CA's identifier\\
    $\mathtt{Sign}_{\mathtt{sk}_{\mathtt{CA}}}($ED$)$\\
    ...\\
  };
  
  \node[mycontainer, fit=(user)] {};
  \node[mycontainer, fit=(cert)] {};
  \draw [->] (user) to (cert);
  \draw [-] (ca) to (fakeN);
  \draw [->] (fakeN) to (fake);
  
\end{tikzpicture}
\end{center}

\begin{center}
CA certifies: $\mathtt{pk}_{\mathtt{E}}$
is indeed the public key of the entity E.
\end{center}
\end{frame}


\begin{frame}{Chain of trust}
\begin{center}
%\scalebox{0.9}{
\begin{tikzpicture}

\node (root) [shape=rectangle,draw,label=above:{Root CA}] {
            \begin{tabular}{c}
                Root CA's name\\
                \hline
                Root CA's public key\\
                \hline
                Root CA's signature\\
            \end{tabular}
        };
        
\node (int) [shape=rectangle,draw,below=of root.south east,label=above:{Intermediate CA}] {
            \begin{tabular}{c}
                Int. CA's name\\
                \hline
                Int. CA's public key\\
                \hline
                Issuer's name (Root CA)\\
                \hline
                Issuer's signature\\
            \end{tabular}
        };
        
\node (end) [shape=rectangle,draw,right=of int.south east,label=above:{End-entity CA}] {
            \begin{tabular}{c}
                Owner's name\\
                \hline
                Owner's public key\\
                \hline
                Issuer's name (Int. CA)\\
                \hline
                Issuer's signature\\
            \end{tabular}
        };        

\node[right = of root.south east, label=right:{{\color{darkviolet} sign}}] (sign) {};

\node[left = of int.north west, label=left:{{\color{olivegreen} reference}}] (sign) {};

\node[left = of root.west, label=below: {{\color{blue} self-sign}}] {};

%\node[left = of end.west]
%{{\color{darkviolet} sign}};
%
%\node[below = of int.south east, label=below:{{\color{darkviolet} sign}}] (sign) {};

% root to root
% root to int      
\draw[->,line width=1pt,blue, bend right] (root.west) to (root.south west);

% root to int      
\draw[->,line width=1pt,darkviolet, bend left] (root.south east) to (int.north east);
\draw[->,olivegreen,line width=1pt, bend left] (int.north west) to (root.south west);

% int to end
\draw[->,line width=1pt,darkviolet, bend left] (int.south east) to (end.north west);
\draw[->,olivegreen,line width=1pt, bend left] (end.south west) to (int.south west);

\end{tikzpicture}
%} %scalebox ends here
\end{center}
\end{frame}



\begin{frame}{Public key infrastructure}
\begin{exampleblock}{Revocation of certificates}
\begin{itemize}
\item Compromised certificates are
\textbf{revoked} by the issuing CA.
\item CA adds revoked certificates to its
\textbf{certificate revocation list (CRL)}.
\item CA publishes updated CRL
\textasciitilde every 24 hours.
\end{itemize}
\end{exampleblock}
\end{frame}

\begin{frame}{Public key infrastructure}

\begin{alertblock}{Problem: single point of failure}
\begin{itemize}
\item Corrupt CA $=$ illegitimate
certificates.
\item Single CA corrupt $=$ PKI's failure.
%\item Corruption of a single CA implies
%failure of PKI.
\end{itemize}
\end{alertblock}

\pause

\begin{exampleblock}{Possible countermeasure}
\begin{itemize}
\item Store certificates and CRL
in an external ledger.
\item What kind of ledger?
\end{itemize}
\end{exampleblock}

\end{frame}


\begin{frame}{Blockchain}

\begin{exampleblock}{Definition}
\begin{itemize}
\item A public, transparent,
append-only ledger.
%\pause
\item Created by members of a peer-to-peer
network.
\item Immutable and unforgeable records (\textbf{blocks}).
%Records \mg{blocks} stored in the ledger are immutable
%and unforgeable
%\item FunFactHere?
\end{itemize}
\end{exampleblock}

\end{frame}



\begin{frame}{Blockchain}

\begin{exampleblock}{Structure}
\begin{itemize}
\item \textbf{Transaction}: atomic event
allowed by the blockchain protocol (`Alice sends Bob $0.1$ BTC', `CA issues a certificate').
\item Transactions are \textbf{validated} and \textbf{broadcasted}
throughout the network.
\item Validated transactions are
stored in \textbf{blocks}.
\item Blocks are linked together,
forming a \textbf{chain}.
\item \textbf{Consensus process}.
\end{itemize}
\end{exampleblock}

\end{frame}


\begin{frame}{Blockchain structure}

\begin{center}
\begin{tikzpicture}[mymatrix/.style={matrix of nodes, nodes=typetag, row sep=1em},
  mycontainer/.style={draw=gray, inner sep=1ex},
  typetag/.style={draw=gray, inner sep=1ex, anchor=west},
  title/.style={draw=none, color=gray, inner sep=0pt}
  ]
  \matrix[mymatrix] (block) {
    |[title]|Block\\
    hash of the previous block\\
    creator's of the block (\textbf{miner}) ID\\
    set of transactions\\
    ...\\
  };
  
  
  \node[mycontainer, fit=(block)] {};

%              up---dummy			dummy3  up2--
%               |                          |
%               |                          |
%  dumll-duml--down dummy2		dummy4--down2

% LHS

  \node[below = of block.north west] (dummy) {};
  \node[left = of dummy.west] (up) {};
  \node[above = of block.south west] (dummy2) {};
  \node[left = of dummy2.west] (down) {};
  
  \draw[-] (dummy) to (up.east);
  \draw[-] (up.east) to (down.east);
  
  \node[left = of dummy2.west] (duml) {};
  \node[left = of duml.west] (dumll) {};
  
  \draw[-] (dumll.west) to (down.east)  {};
  
  
% RHS
  \node[below = of block.north east] (dummy3) {};
  \node[right = of dummy3.east] (up2) {};
  \node[above = of block.south east] (dummy4) {};
  \node[right = of dummy4.east] (down2) {};
  
  \draw[-] (dummy4) to (down2.west);
  \draw[-] (down2.west) to (up2.west);
  
  \node[right = of up2.east] (dumnono) {};
%  \node[right = of dumnono.east] (bzzz) {};
  \draw[-] (up2.west) to (dumnono.east);
  
\end{tikzpicture}
\end{center}

\end{frame}


\begin{frame}{Scenario}
	\begin{exampleblock}{Current scenario}
		user:
		\begin{enumerate}
			\item connects to a website
			\item browser verifies identity of webserver using PKI
		\end{enumerate}
	\end{exampleblock}
	
	\begin{alertblock}{Future scenario}
		user:
		\begin{enumerate}
			\item connects to a website
			\item browser verifies identity of webserver using PKI
			\item browser verifies identity if webserver using Blockchain
		\end{enumerate}
	\end{alertblock}
\end{frame}





%\begin{frame}{Blockchain structure}
%
%\begin{center}
%\begin{tikzpicture}
%
%  \node[draw,rectangle,
%fill=white, label=below:{Genesis block}] (gb) {Block 0};
%
%  \node[draw,rectangle,
%fill=white, right = of gb] (b1) {Block 1};
%
%%  \node[right = of b1.east, label=below:{{\color{white} XX}...{\color{white} XX}}] (dots) {};
%
%  \node[right = of b1.east] (dots) {{\color{white} X}$\ldots${\color{white} X}};
%  
%    \node[draw,rectangle,
%fill=white, right = of dots] (bl) {Block $n$};
%
%  \node[right = of bl.east] (dummy) {};
%  
%  \node[above = of dummy.north] (dummy2) {};
%%  
% \node[below = of dummy.north] (dummy3) {};
%  
%\node[draw,rectangle,
%fill=white, right = of dummy2] (nb1) {Block?};
%
%\node[draw,rectangle,fill=white, right = of dummy3] (nb2) {Block???};
%
%%   \node[draw,rectangle,
%%fill=white, right = of bl.north east](nb1) {Block?};
%%
%%   \node[draw,rectangle,
%%fill=white, right = of bl.south east](nb2) {Block???};
%  
%  
%  \draw [->] (gb) to (b1);
%  \draw [->] (b1) to (dots);
%  \draw [->] (dots) to (bl);
%  \draw [->] (bl) to (nb1);
%  \draw [->] (bl) to (nb2);
%  
%\end{tikzpicture}
%\end{center}
%
%\begin{itemize}
%\item sasa
%\item sese
%\item bzzz
%\end{itemize}
%
%\end{frame}



