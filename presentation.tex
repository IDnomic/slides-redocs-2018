\PassOptionsToPackage{subsection=false}{beamerouterthememiniframes}
% the above removes the subsection bar

\documentclass[11pt, aspectratio=169]{beamer}
\usetheme{Amsterdam}
\setbeamercolor{subsection in head/foot}{fg=White}
%\usetheme{CEA}

\usepackage{lmodern}
\usepackage{multirow}
\usepackage{fontawesome}

\usepackage[T1]{fontenc}

\usepackage[utf8]{inputenc}

\usepackage{graphicx} 

%\usepackage{hyperref}

\usepackage{tikz}
\usetikzlibrary{positioning,patterns,matrix,calc,arrows,shapes,fit,decorations.pathmorphing}


\usepackage{amsmath}
\usepackage{amsfonts}
\usepackage{amssymb}
\usepackage{color}
\newtheorem{thm}{Theorem}
\newtheorem{cor}{Corollary}

\usepackage{algorithm}
\usepackage{algpseudocode}

\usepackage{xspace}
\usepackage{hyperref}

\newcommand{\msl}{\{\hspace*{-0.1cm}|}
\newcommand{\msr}{|\hspace*{-0.1cm}\}}

\definecolor{olivegreen}{RGB}{85, 107, 47}
\definecolor{purple}{RGB}{128, 0, 128}
\definecolor{darkviolet}{RGB}{148, 0, 211}
\newcommand{\mg}[1]{{\color{magenta} #1}}
\newcommand{\cyan}[1]{{\color{cyan} #1}}
\newcommand{\olive}[1]{{\color{olivegreen} #1}}
\newcommand{\red}[1]{{\color{red} #1}}
\newcommand{\green}[1]{{\color{green} #1}}


\renewcommand{\algorithmiccomment}[1]{\hfill\  #1}
\renewcommand{\algorithmicrequire}{\textbf{Input:}}
\renewcommand{\algorithmicensure}{\textbf{Output:}}



%\title{Enhancing public key infrastructure with blockchain}

\title{Towards a decentralized identity management solution based on blockchain --- proof of concept}

%\institute{institute/whatever}

\date{}

%pages enumeration
\setbeamertemplate{footline}[text line]{%
  \parbox{\linewidth}{\vspace*{-8pt}
  REDOCS 2018 \hfill\insertframenumber / \inserttotalframenumber}}
\setbeamertemplate{navigation symbols}{}


%\addtobeamertemplate{frametitle}{\vskip-1.5ex}{}



\AtBeginSection[]
{
  \begin{frame}<beamer>
    \tableofcontents[currentsection]
  \end{frame}
}

\beamertemplatenavigationsymbolsempty
\begin{document}
%\beamertemplatenavigationsymbolsempty

\author{%
\begin{tabular}{rl}
Fabien Charmet & T\'{e}l\'{e}com SudParis, Institut Mines-T\'{e}l\'{e}com,\\
& CNRS Samovar UMR 5157\\
Maxime Montoya & Univ. Grenoble Alpes, CEA, LETI, DACLE\\
Mathieu Valois & Normandie Univ, UNICAEN, ENSICAEN,\\
& CNRS, GREYC\\
Wojciech Wide\l{} & Univ Rennes, INSA Rennes, CNRS, IRISA\\
\end{tabular}
%\vspace{2mm}
\center
26 October 2018}


%\maketitle

\begin{frame}

\titlepage

\vspace{-19mm}
\begin{center}
\begin{tabular}{l @{\hspace{35mm}} r}
\includegraphics[scale=0.14]{redocs_logo2}
&
\includegraphics[scale=0.14]{idnomicLogo} 
\end{tabular} 
\end{center}
\end{frame}


\begin{frame}{Context}

\begin{itemize}
\item Goal
\begin{itemize}
\item Enhancing the trust inside a PKI by adding a verification layer.
\item End users can verify certificate in the blockchain.
\item Separate chain of trust.
\end{itemize}
\item Prerequisites
\begin{itemize}
\item Private blockchain.
\item Private miners.
\item Permission-based transactions.
\end{itemize}
\item Threat model: the attacker can either
\begin{itemize}
\item compromise the CA and generate legitimate certificates, or
\item compromise a node in the blockchain to insert transactions.
\end{itemize}
\end{itemize}

\end{frame}


\begin{frame}{Outline}
\tableofcontents
\end{frame}


\section[sec1]{Background on PKI and blockchains}

\begin{frame}
overview of PKI
\end{frame}


\section[Blockchains for PKI]{How blockchains could enhance PKI}

\begin{frame}{PKI}
	\begin{alertblock}{Issues}
		\begin{itemize}
			\item No way to know if CA is corrupted
			\item CA producing certificates for domains they don't own (Iran with Google)
			\item some web browsers don't check for certification revocation
		\end{itemize}
	\end{alertblock}

	\begin{exampleblock}{Blockchain}
		\begin{itemize}
			\item adds another channel to check for certificate validity
			\item \textbf{transparency} and \textbf{traceability}
			\item Secure distributed log that cannot be altered
			\item The whole chain of trust is stored
			\item revocation lists are all stored in the same place
		\end{itemize}
	\end{exampleblock}
\end{frame}

\begin{frame}{Applications}

	\begin{exampleblock}{Web browsing}
		\begin{itemize}
			\item Privacy and confidentiality issue: are visited websites what they pretend to be?
			\item Millions of certificates, with variable lifetime
		\end{itemize}
	\end{exampleblock}
	
	\begin{alertblock}{Connected cars}
		\begin{itemize}
			\item Safety issue: connected or even autonomous cars might need to check that the surrounding cars are legitimate
			\item Thousands of certificates, with a one-week lifetime
		\end{itemize}
	\end{alertblock}

%\emph{In both cases, all certificates and CRLs would be stocked together in a blockchain.}

\end{frame}

\section[Possible solutions]{Possible solutions}


\begin{frame}{Ethereum smart contracts}

\emph{figure to add}
\emph{citation to add }
%https://ieeexplore.ieee.org/document/8406325

\begin{itemize}
\item Based on the Ethereum blockchain
\item Each certification authority (CA) has \mg{smart contracts} that store a list of issued certificates and a revocation list
\item Specific format for certificates: \mg{hybrid certificates}
%\vspace{2mm}
\item Pros:
\begin{itemize}
\item Scalability: No size limit, no need to scroll the whole blockchain
\end{itemize}
\item Cons:
\begin{itemize}
\item Hard to use and to modify, not generic
\item Hybrid certificates
\item Transaction fees
\end{itemize}

\end{itemize}

%\begin{center}
%\includegraphics{figs/article_pki_blockchain.png}
%\end{center}

\end{frame}





\begin{frame}{Data fields in Bitcoin-based blockchains}

\begin{itemize}
\item \mg{OP\_RETURN} field that can contain data
\item Several blockchains could be used, such as Bitcoin or Namecoin
%\vspace{2mm}
\item Pros:
\begin{itemize}
\item Easy to implement, no hard fork required
\end{itemize}
\item Cons:
\begin{itemize}
\item No permissions: anyone can mine
\item Bitcoin: small size of the field (80 bytes), only fingerprints of certificates
\item Transaction fees
\item No easy revocation protocol
\item Burns bitcoins: they might not be used afterwards
\item The whole blockchain has to be read each time
\end{itemize}

\end{itemize}

\end{frame}





\begin{frame}{New blockchain}
\emph{here we describe the main points of our solution}

\begin{itemize}
\item Creation of a new blockchain with the \mg{Multichain} tool
\item Pros:
\begin{itemize}
\item Very customizable
\item Free transactions
\item Data of any length
\item Permission management
\end{itemize}
\item Cons:
\begin{itemize}
\item None :)
\end{itemize}


\end{itemize}
\end{frame}




%\section[Storing certificates on a ledger]{Storing certificates and
certificate revocation lists on a blockchain ledger}

\begin{frame}
task: enhance PKI by storing certificates
and CRL
in a blokchain ledger

recall the assumptions/threat model
\end{frame}

\subsection[subsec1]{Possible approaches}

\begin{frame}
possible solutions:

- ethereum's smart contracts

- bitcoin's op\_return

- multichain
\end{frame}

\subsection[subsec2]{Our solution}

\begin{frame}
frame
\end{frame}

\section[Our organization]{Our organization}
\begin{frame}
	\frametitle{Our organization}
	\begin{center}
	\includegraphics[scale=0.4]{figs/trello.png}
	\includegraphics[scale=0.4]{figs/slack.png}
	\includegraphics[scale=0.1]{figs/Octocat.jpg}
	\end{center}
\end{frame}

\section[Multichain-based certificate verification]{Multichain-based certificate management}
\begin{frame}
\frametitle{Multichain-based certificate management}
\begin{alertblock}{Multichain}
	\begin{itemize}
		\item fork of the Bitcoin source code
		\item hugely simplifies private Blockchains creation and management
		\item lot of settings available
		\item node permission control
		\item arbitrary-sized data field in transactions
		\item very well documented
	\end{itemize}
\end{alertblock}
\end{frame}

\begin{frame}
	\frametitle{Comparison}
\end{frame}


\begin{frame}
	\frametitle{Design}
	\begin{center}
		\begin{tikzpicture}
		[node distance=1cm and 4cm]
		\node[label=above:{CA}](CA) {{\LARGE \faUser}};
		\pause
		\draw[-] (CA) to (d3);
		\node[below = of r.south] (d3) {sign};
		\node[label=below:{Cert}, below = of d3.south] (cacert) {{\Huge \faFileTextO}};				\draw[->] (d3) to (cacert);
		
		\pause
		
		\node[right=of CA.east, label=above:{secure connection}](secure){};
		\node[label=above:{Miner}, right = of secure.east](Miner){{\LARGE \faUser}};
		\draw[->] (CA) to (Miner);
		\pause
		\draw[-] (Miner) to (d4);
		\node[below = of r.south, below = of Miner.south] (d4) {produce};
		\node[label=below:{Transaction}, below = of d4.south] (trans) {{\Huge \faFileTextO}};
		\draw[->] (d4) to (trans);
		
		\end{tikzpicture}
	\end{center}
\end{frame}

\begin{frame}
	\frametitle{Scenario}
	\begin{alertblock}{}
		\begin{enumerate}
			\item final user visits a website with web browser
			\item classical identity verification is used (PKI)
			\item browser plug-in installed on the user browser
			\item local daemon is running, waiting for queries
			\item plugin-in retrieves certificates, asking to daemon if such a certificate is valid
			\item displays whether certificates should be trusted or not
		\end{enumerate}
	\end{alertblock}
\end{frame}

\section[Demonstration]{Demo}

\begin{frame}
	\frametitle{Demo}
	\begin{center}
		\includegraphics[scale=0.4]{figs/green_certs.png}
		\includegraphics{figs/green_tick.png}
	\end{center}
\end{frame}

\begin{frame}
	\frametitle{Demo}
	\begin{center}
		\includegraphics[scale=0.4]{figs/red_certs.png}
		\includegraphics{figs/red_mark.png}
	\end{center}
\end{frame}


\end{document}